\documentclass[12pt]{article} 
\usepackage[utf8]{inputenc}
\usepackage{geometry}
\geometry{letterpaper, margin=0.5in}
\usepackage{graphicx} 
\usepackage{parskip}
\usepackage{booktabs}
\usepackage{array} 
\usepackage{paralist} 
\usepackage{verbatim}
\usepackage{subfig}
\usepackage{fancyhdr}
\usepackage{sectsty}
\usepackage[shortlabels]{enumitem}

\pagestyle{fancy}
\renewcommand{\headrulewidth}{0pt} 
\lhead{}\chead{}\rhead{}
\lfoot{}\cfoot{\thepage}\rfoot{}


%%% ToC (table of contents) APPEARANCE
\usepackage[nottoc,notlof,notlot]{tocbibind} 
\usepackage[titles,subfigure]{tocloft}
\renewcommand{\cftsecfont}{\rmfamily\mdseries\upshape}
\renewcommand{\cftsecpagefont}{\rmfamily\mdseries\upshape} %

\usepackage{amsmath}
\usepackage{amssymb}
\usepackage{mathtools}
\usepackage{empheq}
\usepackage{xcolor}

\usepackage{tikz}
\usepackage{pgfplots}
\usepackage{tikz-cd}
\pgfplotsset{compat=1.18}

\newcommand{\ans}[1]{\boxed{\text{#1}}}
\newcommand{\vecs}[1]{\langle #1\rangle}
\renewcommand{\hat}[1]{\widehat{#1}}
\newcommand{\F}[1]{\mathcal{F}(#1)}
\renewcommand{\P}{\mathbb{P}}
\newcommand{\R}{\mathbb{R}}
\newcommand{\E}{\mathbb{E}}
\newcommand{\Z}{\mathbb{Z}}
\newcommand{\N}{\mathbb{N}}
\newcommand{\Q}{\mathbb{Q}}
\newcommand{\ind}{\mathbbm{1}}
\newcommand{\qed}{\quad \blacksquare}
\newcommand{\brak}[1]{\left\langle #1 \right\rangle}
\newcommand{\bra}[1]{\left\langle #1 \right\vert}
\newcommand{\ket}[1]{\left\vert #1 \right\rangle}
\newcommand{\abs}[1]{\left\vert #1 \right\vert}
\newcommand{\mfX}{\mathfrak{X}}
\newcommand{\ep}{\varepsilon}

\newcommand{\Ec}{\mathcal{E}}
\newcommand{\sub}{\subseteq}
\newcommand{\st}{\text{ s.t. }}
\renewcommand{\div}{\vspace*{10pt}\hrule\vspace*{10pt}}

\newcommand*{\tbf}[1]{\ifmmode\mathbf{#1}\else\textbf{#1}\fi}

\usepackage{tcolorbox}
\tcbuselibrary{breakable, skins}
\tcbset{enhanced}
\newenvironment*{tbox}[2][gray]{
    \begin{tcolorbox}[
parbox=false,
        colback=#1!5!white,
        colframe=#1!75!black,
        breakable,
        title={#2}
    ]}
    {\end{tcolorbox}}

\newenvironment*{proof}[1][blue]{
    \begin{tcolorbox}[
    parbox=false,
        colback=#1!5!white,
        colframe=#1!75!black,
        coltext=#1,
        breakable
    ]}
    {\end{tcolorbox}}

\title{APMA 1930X: Homework 3}
\author{Milan Capoor}
\date{Oct 18}

\begin{document}
\maketitle

1. (Kelly Criterion)  Suppose you are playing a betting game with the initial stake $S_0 = x_0 > _0$. Let $S_n$ be your wealth at the end of the $n$-th round. Let $Y_{n+1}$ be the amount you bet at the $(n + 1)$-th round. Your winning per-unit-stake on the $(n + 1)$-th round of betting can be represented by $X_{n+1}$, where $\{X_1, X_2, \dots\}$ are iid random  variables with
\[P(X_{n+1} = a) = p, P(X_{n+1} = -b) = q = 1 - p\]

In other words, you will win the amount $aY_{n+1}$ with probability $p$, and lose the amount $bY_{n+1}$ with probability $q$. The dynamics of the wealth process is given by
\[S_{n+1} = S_n + X_{n+1} Y_{n+1},\quad n \geq 0\]
Here we should assume that $a, b > 0$ and the game is in favor of you,
that is,
\[\E[X_{n+1}] = ap - bq > 0\]
and
\[0 \leq Y_{n+1} \leq \frac{S_n}{b}\]

The upper bound on $Y_{n+1}$ is to ensure that the loss will not exceed $S_n$.

Suppose the game ends at the $N$-th round. We wish to find an optimal betting strategy $\{Y_1^*, Y_2^*, \dots, Y_{N}^*\}$ so as the maximize the expected utility $\E[\log S_N]$

\tbf{Hint:} Mimic Example 2.9. Verify the the value function takes the form 
\[v_n(x) = \log x + (N - n)c\]
for some constant $c$. 

    \color{blue}
        Let $V_n(x) = \sup \E[\log S_N \; | \; S_n = x]$, i.e. the maximum expected utility at time $n$ given that the wealth is $x$. 

        Clearly, 
        \[V_N(x) = \sup \E[\log S_n \; | \; S_N = x] = \sup \E[\log x] = \log x\]

        Now, per usual, suppose that (with $0 \leq n \leq N - 1$), $S_n = x$ and we bet $Y_{n+1} = y$ at the $n+1$-th round but then bet optimally for $n+2, \dots, N$.

        If we win (with probability $p$), then $S_{n+1} = x + ay$ and if we lose (with probability $q$), then $S_{n+1} = x - by$. In the case we win, our maximum utility is just $V_{n+1}(x + ay)$ and in the case we lose, our maximum utility is $V_{n+1}(x - by)$.

        Hence, 
        \[\E[V_{n+1}(S_{n+1})] = pV_{n+1}(x + ay) + qV_{n+1}(x - by)\]

        So to find the optimal betting strategy, we introduce the DPE 
        \[\begin{cases}
            V_n(x) = \sup_{0 \leq y \leq x} [pV_{n+1}(x + ay) + qV_{n+1}(x - by)]\\ 
            V_N(x) = \log x
        \end{cases}\]

        Let's calculate a few cases:
        \begin{align*}
            V_{N-1}(x) &= \sup_{0 \leq y \leq x} [pV_N(x + ay) + qV_N(x - by)]\\ 
                &= \sup_{0 \leq y \leq x} [p\log(x + ay) + q\log(x - by)]\\ 
        \end{align*}

        And 
        \begin{align*}
            0 &= \frac{d}{dy}\left[ p\log(x + ay) + q\log(x - by)\right]\\ 
                &= \frac{ap}{x + ay} - \frac{bq}{x - by}\\
                &\implies y = \frac{xap - xbq}{ab(p+q)}
        \end{align*}

        So 
        \[y^* = \frac{x(ap - bq)}{ab}\]

        Let $c^* = \frac{ap - bq}{ab}$. Then
        \begin{align*}
            V_{N-1}(x) &= p\log(x + ac^*x) + q\log(x - bc^*x)\\ 
                &= p\log(x(1 + ac^*)) + q\log(x(1 - bc^*))\\ 
                &= p\log x + p\log(1 + ac^*) + q\log x + q\log(1 - bc^*)\\ 
                &= p\log x + p\log(1 + ac^*) + (1 - p)\log x + q\log(1 - bc^*)\\
                &= \log x + p\log(1 + ac^*) + q\log(1 - bc^*)
        \end{align*}

        But the last two terms are constants so $V_{n-1}(x) = \log x + c$ for $c = p\log(1 + ac^*) + q\log(1 - bc^*)$.

        Similarly, 
        \begin{align*}
            V_{N-2}(x) &= pV_{N-1}(x + ay^*) + qV_{N-1}(x - by^*)\\ 
                &= p(\log(x + axc^*) + c) + q(\log(x - bxc^*) + c)\\
                &= p\log x + p\log(1 + ac^*) + pc + q\log x + q\log(1 - bc^*) + qc\\
                &= \log x + (p + q)c + p\log(1 + ac^*) + q\log(1 - bc^*)\\ 
                &= \log x + c + c\\ 
                &= \log x + 2c
        \end{align*}

        Suppose that $V_{n+1}(x) = \log x + (N - n - 1)c$. Then
        \begin{align*}
            V_n(x) &= \sup_{0 \leq y \leq x} [pV_{n+1}(x + ay) + qV_{n+1}(x - by)]\\ 
                &= \sup_{0 \leq y \leq x} [p\log(x + ay) + p(N - n - 1)c + q\log(x - by) + q(N - n - 1)c]\\
        \end{align*}

        Taking the derivative with respect to $y$, we get
        \begin{align*}
            0 &= \frac{d}{dy} [p\log(x + ay) + p(N - n - 1)c + q\log(x - by) + q(N - n - 1)c]\\ 
            &= \frac{ap}{x + ay} - \frac{bq}{x - by}\\ 
            &\implies y = x\frac{ap - bq}{ab(p+q)} = xc^*
        \end{align*}
        so 
        \begin{align*}
            V_n(x) &= [p\log(x + ay^*) + p(N - n - 1)c + q\log(x - by^*) + q(N - n - 1)c]\\ 
                &= p\log(x + axc^*) + p(N - n - 1)c + q\log(x - bxc^*) + q(N - n - 1)c\\
                &= (p + q)\log x + p\log(1 + ac^*) + q\log x \log(1 - bc^*) + (p + q)(N - n - 1)c\\
                &= \log x + c + (N - n - 1)c\\
                &= \log x + (N - n)c
        \end{align*} 

        Hence, by induction, $V_n(x) = \log x + (N - n)c$ for 
        \[c = p\log(1 + \frac{ap - bq}{b}) + q\log(1 - \frac{ap - bq}{a})\] 
        
        So the optimal betting strategy is $Y_n^* = \frac{S_n(ap - bq)}{ab}$.
       
    \color{black}

\end{document}