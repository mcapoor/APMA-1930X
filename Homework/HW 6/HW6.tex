\documentclass[12pt]{article}
\usepackage[utf8]{inputenc}
\usepackage{geometry}
\geometry{letterpaper, margin=0.25in}
\usepackage{graphicx} 
\usepackage{parskip}
\usepackage{booktabs}
\usepackage{array} 
\usepackage{paralist} 
\usepackage{verbatim}
\usepackage{subfig}
\usepackage{fancyhdr}
\usepackage{sectsty}
\usepackage[shortlabels]{enumitem}

\pagestyle{fancy}
\renewcommand{\headrulewidth}{0pt} 
\lhead{}\chead{}\rhead{}
\lfoot{}\cfoot{\thepage}\rfoot{}

%%% ToC (table of contents) APPEARANCE
\usepackage[nottoc,notlof,notlot]{tocbibind} 
\usepackage[titles,subfigure]{tocloft}
\renewcommand{\cftsecfont}{\rmfamily\mdseries\upshape}
\renewcommand{\cftsecpagefont}{\rmfamily\mdseries\upshape} %

\usepackage{amsmath}
\usepackage{amssymb}
\usepackage{mathtools}
\usepackage{empheq}
\usepackage{xcolor}
\usepackage{bbm}
\usepackage{tikz}
\usepackage{pgfplots}
\usepackage{tikz-cd}
\pgfplotsset{compat=1.18}

\newcommand{\ans}[1]{\boxed{\text{#1}}}
\newcommand{\vecs}[1]{\langle #1\rangle}
\renewcommand{\hat}[1]{\widehat{#1}}

\renewcommand{\P}{\mathbb{P}}
\newcommand{\R}{\mathbb{R}}
\newcommand{\E}{\mathbb{E}}
\newcommand{\Z}{\mathbb{Z}}
\newcommand{\N}{\mathbb{N}}
\newcommand{\Q}{\mathbb{Q}}
\newcommand{\C}{\mathbb{C}}

\newcommand{\ind}{\mathbbm{1}}
\newcommand{\qed}{\quad \blacksquare}

\newcommand{\brak}[1]{\left\langle #1 \right\rangle}
\newcommand{\bra}[1]{\left\langle #1 \right\vert}
\newcommand{\ket}[1]{\left\vert #1 \right\rangle}

\newcommand{\abs}[1]{\left\vert #1 \right\vert}
\newcommand{\mfX}{\mathfrak{X}}
\newcommand{\ep}{\varepsilon}

\newcommand{\Ec}{\mathcal{E}}
\newcommand{\A}{\mathcal{A}}
\newcommand{\F}{\mathcal{F}}
\newcommand{\Cc}{\mathcal{C}}
\newcommand{\B}{\mathcal{B}}
\newcommand{\M}{\mathcal{M}}
\newcommand{\X}{\chi}
\renewcommand{\L}{\mathcal{L}}

\newcommand{\sub}{\subseteq}
\newcommand{\st}{\text{ s.t. }}
\newcommand{\card}{\text{card }}
\renewcommand{\div}{\vspace*{10pt}\hrule\vspace*{10pt}}
\newcommand{\surj}{\twoheadrightarrow}
\newcommand{\inj}{\hookrightarrow}
\newcommand{\biject}{\hookrightarrow \hspace{-8pt} \rightarrow}
\renewcommand{\bar}[1]{\overline{#1}}
\newcommand{\overcirc}[1]{\overset{\circ}{#1}}
\newcommand{\diam}{\text{diam }}

\renewcommand{\Re}{\text{Re}\,}
\renewcommand{\Im}{\text{Im}\,}
\newcommand{\sign}{\text{sign}\,}

\newcommand*{\tbf}[1]{\ifmmode\mathbf{#1}\else\textbf{#1}\fi}

\usepackage{tcolorbox}
\tcbuselibrary{breakable, skins}
\tcbset{enhanced}
\newenvironment*{tbox}[2][gray]{
    \begin{tcolorbox}[
        parbox=false,
        colback=#1!5!white,
        colframe=#1!75!black,
        breakable,
        title={#2}
    ]}
    {\end{tcolorbox}}

\newenvironment*{exercise}[1][red]{
    \begin{tcolorbox}[
        parbox=false,
        colback=#1!5!white,
        colframe=#1!75!black,
        breakable
    ]}
    {\end{tcolorbox}}

\newenvironment*{proof}[1][blue]{
\begin{tcolorbox}[
    parbox=false,
    colback=#1!5!white,
    colframe=#1!75!black,
    breakable
]}
{\end{tcolorbox}}

\title{APMA 1930X: Homework 6}
\author{Milan Capoor}
\date{13 December 2024}

\begin{document}
\maketitle

Fix an arbitrary time $T > 0$. Let $r, \theta, \sigma$ be given constants with $\sigma > 0$.
Consider the controlled process
\[\frac{dX_t}{X_t} = (r + \theta \pi_t)\; dt + \sigma \pi_t \; dBt,\quad X_0 = x_0\] 
where $\pi  = {\pi_t : 0 \leq t \leq T}$ is the control. 

Our goal is to maximize the expected power utility $\E[\sqrt{X_T}]$ over all control policies $\pi$. Your task is to
\begin{enumerate}
    \item clearly define the value function;
    \item write down the Hamilton-Jacobi-Bellman (HJB) equation and give some short explanation as to how you arrive at this equation;
    \item solve the HJB equation explicitly;
    \item identify the optimal control policy $\pi^*$ explicitly.
\end{enumerate}

\tbf{Hint:} In order to solve the equation, you can assume that the value function $v(t, x)$ takes the form $f(t)\sqrt{x}$ for some function $f$. Then you can turn the HJB equation (a partial differential equation) into a very simple ordinary differential equation of $f$.

    \color{blue}
        Our goal is to maximize the expected power utility $\E[\sqrt{X_T}]$ over all control policies $\pi$ on $[0, T]$.

        Hence, 
        \[v(t, x) = \sup_{\pi_t} \E\left[\int_t^{T} \sqrt{X_s}\; ds \; | \; X_t = x\right]\]
        and we seek $v(0, x_0)$. 

        Let $\ep > 0$ be a small increment in time. As always, applying an optimal control for $[t + \ep, T]$ and an arbitrary control $[t, t + \ep]$, 
        \[v(t, x) \geq \E\left[\int_t^{t + \ep} \sqrt{X_s}\; ds + v(t+ \ep, X_{t + \ep}) \right]\]

        First consider $\E[v(t + \ep, X_{t + \ep})]$:
        \begin{align*}
            d(v(t, X_t)) &= \partial_t v(t, X_t) \; dt + \partial_x v(t, X_t) \; dX_t + \frac{1}{2} \partial_{xx} v(t, X_t) \; (dX_t)^2\\
            &= \left[\frac{\partial v}{\partial t} + (r + \theta \pi_t)x \frac{\partial v}{\partial x} + \frac{1}{2} \sigma^2 \pi_t^2x^2 \frac{\partial^2 v}{\partial x^2}\right] \; dt + \sigma \pi_t x \frac{\partial v}{\partial x} \; dB_t
        \end{align*}
        with partial derivatives evaluated at $(t, X_t)$.

        Now, 
        \begin{align*}
            v(t + \ep, X_{t + \ep}) - v(t, X_t) &= \int_{t}^{t + \ep} \; dv(s, X_s)\\ 
                &= \int_{t}^{t + \ep} \left[\frac{\partial v}{\partial t} + (r + \theta \pi_s)x \frac{\partial v}{\partial x} + \frac{1}{2} \sigma^2 \pi_s^2x^2 \frac{\partial^2 v}{\partial x^2}\right] \; ds + \int_t^{t + \ep} \sigma \pi_t x\frac{\partial v}{\partial x} \; dB_s
        \end{align*}
        
        And because stochastic integrals have zero expectation, 
        \[\E[v(t + \ep, X_{t, X_t})] - \E[v(t, X_t)] = \E\left[\int_{t}^{t + \ep} \left[\frac{\partial v}{\partial t} + (r + \theta \pi_s) x\frac{\partial v}{\partial x} + \frac{1}{2} \sigma^2 \pi_s^2 x^2 \frac{\partial^2 v}{\partial x^2}\right] \; ds\right] = 0\]

        So we have our HJB equation given by 
        \[\max_{\pi_t} \left[\frac{\partial v}{\partial t} + (r + \theta \pi_t) x\frac{\partial v}{\partial x} + \frac{1}{2} \sigma^2 \pi_t^2 x^2\frac{\partial^2 v}{\partial x^2}\right] = 0\]

        For simplicity, we will assume that $v(t, x) = f(t)\sqrt{x}$ for some function $f$ and verify our solution. 

        With this assumption, the HJB equation becomes
        \begin{gather*}
            f'(t)\sqrt{x} + \frac{(r + \theta \pi_t)x f(t)}{2\sqrt{x}} -  \frac{\sigma^2 \pi_t^2 x^2f(t)}{8x^{3/2}} = 0\\ 
            f'(t)\sqrt{x} + \frac{\sqrt x}{2}(r + \theta \pi_t)f(t) -\frac{1}{8} \sigma^2 \pi_t^2 \sqrt{x} f(t) = 0\\ 
            f'(t) + \left[\frac{1}{2}(r + \theta \pi_t) - \frac{1}{8}\sigma^2 \pi_t^2\right]f(t) = 0
        \end{gather*}
        with terminal condition $v(T, x) = \sqrt{x} \implies f(T) = 1$.

        Letting $\beta = \frac{1}{2}(r + \theta \pi_t) - \frac{1}{8}\sigma^2 \pi_t^2$, we have the simple ODE 
        \[f'(t) + \beta f(t) = 0 \implies f(t) = Ce^{-\beta t}\]
        
        Using the terminal condition, $f(T) = 1 \implies C = e^{\beta T}$. Hence,
        \[f(t) = e^{\beta(T - t)} \implies \boxed{v(t, x) = e^{\beta(T - t)}\sqrt{x}}\] 

        Finally, the optimal control is given by 
        \begin{gather*}
            \max_{\pi_t} \beta = \max_{\pi_t}\left[\frac{1}{2}(r + \theta \pi_t) - \frac{1}{8}\sigma^2 \pi_t^2\right]\\ 
            \frac{d}{d\pi}\left[\frac{1}{2}(r + \theta \pi) - \frac{1}{8}\sigma^2 \pi^2\right] = 0\\
            \frac{1}{2}\theta - \frac{\sigma^2}{4} \pi = 0\\ 
            \boxed{\pi^* = \frac{2\theta}{\sigma^2}}
        \end{gather*}

        So, at last, the maximized expected power utility is 
        \begin{align*}
            v(0, x_0) &= \sqrt{x_0} \exp\left[\left(\frac{1}{2}(r + \theta \pi^*) - \frac{1}{8}\sigma^2 (\pi^*)^2\right)(T)\right]\\ 
            &= \sqrt{x_0}\exp\left[\left(\frac{1}{2}(r + \theta \cdot \frac{2\theta}{\sigma^2}) - \frac{1}{8}\sigma^2 \cdot \frac{4\theta^2}{\sigma^4}\right)(T)\right]\\ 
            &= \boxed{\sqrt{x_0}\exp\left[T\left(\frac{r}{2} + \frac{\theta^2}{\sigma^2} - \frac{\theta^2}{2\sigma^2}\right)\right]}
        \end{align*}

        
    \color{black}

\end{document}